%%%%%%%%%%%%%%%%%%%%%%%%%%%%%%%%%%%%%%%%%%%%%%%%%%%%%%%%%%%%%%%%%%%%%%%%
%%%%%%%%%%%%%%%%%%%%%% Simple LaTeX CV Template %%%%%%%%%%%%%%%%%%%%%%%%
%%%%%%%%%%%%%%%%%%%%%%%%%%%%%%%%%%%%%%%%%%%%%%%%%%%%%%%%%%%%%%%%%%%%%%%%

%%%%%%%%%%%%%%%%%%%%%%%%%%%%%%%%%%%%%%%%%%%%%%%%%%%%%%%%%%%%%%%%%%%%%%%%
%% NOTE: If you find that it says                                     %%
%%                                                                    %%
%%                           1 of ??                                  %%
%%                                                                    %%
%% at the bottom of your first page, this means that the AUX file     %%
%% was not available when you ran LaTeX on this source. Simply RERUN  %%
%% LaTeX to get the ``??'' replaced with the number of the last page  %%
%% of the document. The AUX file will be generated on the first run   %%
%% of LaTeX and used on the second run to fill in all of the          %%
%% references.                                                        %%
%%%%%%%%%%%%%%%%%%%%%%%%%%%%%%%%%%%%%%%%%%%%%%%%%%%%%%%%%%%%%%%%%%%%%%%%

%%%%%%%%%%%%%%%%%%%%%%%%%%%% Document Setup %%%%%%%%%%%%%%%%%%%%%%%%%%%%

% Don't like 10pt? Try 11pt or 12pt
\documentclass[10pt]{article}

% The automated optical recognition software used to digitize resume
% information works best with fonts that do not have serifs. This
% command uses a sans serif font throughout. Uncomment both lines (or at
% least the second) to restore a Roman font (i.e., a font with serifs).
%\usepackage{times}
%\renewcommand{\familydefault}{\sfdefault}

% This is a helpful package that puts math inside length specifications
\usepackage{calc}
\usepackage{comment}

% Simpler bibsection for CV sections
% (thanks to natbib for inspiration)
\makeatletter
\newlength{\bibhang}
\setlength{\bibhang}{1em} %1em}
\newlength{\bibsep}
 {\@listi \global\bibsep\itemsep \global\advance\bibsep by\parsep}
\newenvironment{bibsection}%
        {\begin{enumerate}{}{%
%        {\begin{list}{}{%
       \setlength{\leftmargin}{\bibhang}%
       \setlength{\itemindent}{-\leftmargin}%
       \setlength{\itemsep}{\bibsep}%
       \setlength{\parsep}{\z@}%
        \setlength{\partopsep}{0pt}%
        \setlength{\topsep}{0pt}}}
        {\end{enumerate}\vspace{-.6\baselineskip}}
%        {\end{list}\vspace{-.6\baselineskip}}
\makeatother

% Layout: Puts the section titles on left side of page
\reversemarginpar

%
%         PAPER SIZE, PAGE NUMBER, AND DOCUMENT LAYOUT NOTES:
%
% The next \usepackage line changes the layout for CV style section
% headings as marginal notes. It also sets up the paper size as either
% letter or A4. By default, letter was used. If A4 paper is desired,
% comment out the letterpaper lines and uncomment the a4paper lines.
%
% As you can see, the margin widths and section title widths can be
% easily adjusted.
%
% ALSO: Notice that the includefoot option can be commented OUT in order
% to put the PAGE NUMBER *IN* the bottom margin. This will make the
% effective text area larger.
%
% IF YOU WISH TO REMOVE THE ``of LASTPAGE'' next to each page number,
% see the note about the +LP and -LP lines below. Comment out the +LP
% and uncomment the -LP.
%
% IF YOU WISH TO REMOVE PAGE NUMBERS, be sure that the includefoot line
% is uncommented and ALSO uncomment the \pagestyle{empty} a few lines
% below.
%

%% Use these lines for letter-sized paper
\usepackage[paper=letterpaper,
            %includefoot, % Uncomment to put page number above margin
            marginparwidth=1.2in,     % Length of section titles
            marginparsep=.05in,       % Space between titles and text
            margin=1in,               % 1 inch margins
            includemp]{geometry}

%% Use these lines for A4-sized paper
%\usepackage[paper=a4paper,
%            %includefoot, % Uncomment to put page number above margin
%            marginparwidth=30.5mm,    % Length of section titles
%            marginparsep=1.5mm,       % Space between titles and text
%            margin=25mm,              % 25mm margins
%            includemp]{geometry}

%% More layout: Get rid of indenting throughout entire document
\setlength{\parindent}{0in}

\usepackage[shortlabels]{enumitem}

%% Reference the last page in the page number
%
% NOTE: comment the +LP line and uncomment the -LP line to have page
%       numbers without the ``of ##'' last page reference)
%
% NOTE: uncomment the \pagestyle{empty} line to get rid of all page
%       numbers (make sure includefoot is commented out above)
%
\usepackage{fancyhdr,lastpage}
\pagestyle{fancy}
%\pagestyle{empty}      % Uncomment this to get rid of page numbers
\fancyhf{}\renewcommand{\headrulewidth}{0pt}
\fancyfootoffset{\marginparsep+\marginparwidth}
\newlength{\footpageshift}
\setlength{\footpageshift}
          {0.5\textwidth+0.5\marginparsep+0.5\marginparwidth-2in}
\lfoot{\hspace{\footpageshift}%
       \parbox{4in}{\, \hfill %
                    \arabic{page} of \protect\pageref*{LastPage} % +LP
%                    \arabic{page}                               % -LP
                    \hfill \,}}

% Finally, give us PDF bookmarks
\usepackage{color,hyperref}
\definecolor{darkblue}{rgb}{0.0,0.0,0.3}
\hypersetup{colorlinks,breaklinks,
            linkcolor=darkblue,urlcolor=darkblue,
            anchorcolor=darkblue,citecolor=darkblue}

%%%%%%%%%%%%%%%%%%%%%%%% End Document Setup %%%%%%%%%%%%%%%%%%%%%%%%%%%%


%%%%%%%%%%%%%%%%%%%%%%%%%%% Helper Commands %%%%%%%%%%%%%%%%%%%%%%%%%%%%

% The title (name) with a horizontal rule under it
% (optional argument typesets an object right-justified across from name
%  as well)
%
% Usage: \makeheading{name}
%        OR
%        \makeheading[right_object]{name}
%
% Place at top of document. It should be the first thing.
% If ``right_object'' is provided in the square-braced optional
% argument, it will be right justified on the same line as ``name'' at
% the top of the CV. For example:
%
%       \makeheading[\emph{Curriculum vitae}]{Your Name}
%
% will put an emphasized ``Curriculum vitae'' at the top of the document
% as a title. Likewise, a picture could be included:
%
%   \makeheading[\includegraphics[height=1.5in]{my_picutre}]{Your Name}
%
% the picture will be flush right across from the name.
\newcommand{\makeheading}[2][]%
        {\hspace*{-\marginparsep minus \marginparwidth}%
         \begin{minipage}[t]{\textwidth+\marginparwidth+\marginparsep}%
             {\large \bfseries #2 \hfill #1}\\[-0.15\baselineskip]%
                 \rule{\columnwidth}{1pt}%
         \end{minipage}}

% The section headings
%
% Usage: \section{section name}
\renewcommand{\section}[1]{\pagebreak[3]%
    \hyphenpenalty=10000%
    \vspace{1.3\baselineskip}%
    \phantomsection\addcontentsline{toc}{section}{#1}%
    \noindent\llap{\scshape\smash{\parbox[t]{\marginparwidth}{\raggedright #1}}}%
    \vspace{-\baselineskip}\par}

% An itemize-style list with lots of space between items
\newenvironment{outerlist}[1][\enskip\textbullet]%
        {\begin{itemize}[#1,leftmargin=*]}{\end{itemize}%
         \vspace{-.6\baselineskip}}

% An environment IDENTICAL to outerlist that has better pre-list spacing
% when used as the first thing in a \section
\newenvironment{lonelist}[1][\enskip\textbullet]%
        {\begin{list}{#1}{%
        \setlength{\partopsep}{0pt}%
        \setlength{\topsep}{0pt}}}
        {\end{list}\vspace{-.6\baselineskip}}

% An itemize-style list with little space between items
\newenvironment{innerlist}[1][\enskip\textbullet]%
        {\begin{itemize}[#1,leftmargin=*,parsep=0pt,itemsep=0pt,topsep=0pt,partopsep=0pt]}
        {\end{itemize}}

% An environment IDENTICAL to innerlist that has better pre-list spacing
% when used as the first thing in a \section
\newenvironment{loneinnerlist}[1][\enskip\textbullet]%
        {\begin{itemize}[#1,leftmargin=*,parsep=0pt,itemsep=0pt,topsep=0pt,partopsep=0pt]}
        {\end{itemize}\vspace{-.6\baselineskip}}

% To add some paragraph space between lines.
% This also tells LaTeX to preferably break a page on one of these gaps
% if there is a needed pagebreak nearby.
\newcommand{\blankline}{\quad\pagebreak[3]}
\newcommand{\halfblankline}{\quad\vspace{-0.5\baselineskip}\pagebreak[3]}

% Uses hyperref to link DOI
\newcommand\doilink[1]{\href{http://dx.doi.org/#1}{#1}}
\newcommand\doi[1]{doi:\doilink{#1}}

% For \url{SOME_URL}, links SOME_URL to the url SOME_URL
\providecommand*\url[1]{\href{#1}{#1}}
% Same as above, but pretty-prints SOME_URL in teletype fixed-width font
\renewcommand*\url[1]{\href{#1}{\texttt{#1}}}

% For \email{ADDRESS}, links ADDRESS to the url mailto:ADDRESS
\providecommand*\email[1]{\href{mailto:#1}{#1}}
% Same as above, but pretty-prints ADDRESS in teletype fixed-width font
%\renewcommand*\email[1]{\href{mailto:#1}{\texttt{#1}}}

%\providecommand\BibTeX{{\rm B\kern-.05em{\sc i\kern-.025em b}\kern-.08em
%    T\kern-.1667em\lower.7ex\hbox{E}\kern-.125emX}}
%\providecommand\BibTeX{{\rm B\kern-.05em{\sc i\kern-.025em b}\kern-.08em
%    \TeX}}
\providecommand\BibTeX{{B\kern-.05em{\sc i\kern-.025em b}\kern-.08em
    \TeX}}
\providecommand\Matlab{\textsc{Matlab}}


\newcommand\textline[4][t]{%
  \par\smallskip\noindent\parbox[#1]{.333\textwidth}{\raggedright\texttt{+}#2}%
  \parbox[#1]{.333\textwidth}{\centering#3}%
  \parbox[#1]{.333\textwidth}{\raggedleft\texttt{#4}}\par\smallskip%
}

%%%%%%%%%%%%%%%%%%%%%%%% End Helper Commands %%%%%%%%%%%%%%%%%%%%%%%%%%%

%%%%%%%%%%%%%%%%%%%%%%%%% Begin CV Document %%%%%%%%%%%%%%%%%%%%%%%%%%%%

\begin{document}
\makeheading{Kraig J. Andrews}

\section{Contact Information}

% NOTE: Mind where the & separators and \\ breaks are in the following
%       table.
%
% ALSO: \rcollength is the width of the right column of the table
%       (adjust it to your liking; default is 1.85in).
%
\newlength{\rcollength}\setlength{\rcollength}{1.4in}%
%
\begin{tabular}[t]{@{}p{\textwidth-\rcollength}p{\rcollength}}
%\href{http://www.cse.osu.edu/}%
%     {Department of Computer Science and Engineering} & \\
%\href{http://www.osu.edu/}{The Ohio State University}
666 West Hancock Street   & $+1$~248-798-9388 \\
Detroit, MI 48201     & \email{kraig.andrews@wayne.edu}\\
& \url{kraigjandrews.com}
\end{tabular}

%\section{Objective}

%Insert text here if you want to
%\begin{innerlist}
%\item More information and auxiliary documents can be found at\\\url{http://www.tedpavlic.com/facjobsearch/}
%\end{innerlist}

\section{Research Interests}
Two-dimensional materials, nanotechnology, transition metal dichalcogenides, field-effect transistors, semiconductor physics, materials physics

\section{Education}

\href{http://www.wayne.edu}{\textbf{Wayne State University}},
Detroit, MI \hfill \emph{expected} 2018
\begin{outerlist}

\item[] Ph.D.,
        \href{http://physics.clas.wayne.edu/}
             {Physics}
        \begin{innerlist}
        \item Thesis Topic: ``Intrinsic Channel Properties, Scattering Mechanisms, and Quantum Transport Properties in Transition Metal Dichalcogenides"
        \item Advisor:
              \href{http://www.physics.wayne.edu/~zxzhou/}
                   {Zhixian Zhou, Ph.D}
        \end{innerlist}
\end{outerlist}
\vspace{.1in} 

\href{http://www.wayne.edu}{\textbf{Wayne State University}}, Detroit, MI \hfill 2016
\begin{outerlist}
\item[] M.S.,
        \href{http://physics.clas.wayne.edu/}
             {Physics},
             Feb 2016
\end{outerlist}

\vspace{.1in}
\href{http://www.msu.edu}{\textbf{Michigan State University}},
East Lansing, MI \hfill 2014
\begin{outerlist}
\item[] B.S.,
        \href{http://www.pa.msu.edu/}
             {Physics and Astrophysics (Double Major)}

\end{outerlist}

\section{Research Experience}
\textbf{Graduate Research Assistant} \hfill {2015--Present}
\begin{innerlist}
    
    \item[] \href{http://www.physics.wayne.edu/~zxzhou/}{Nano Fabrication and Electron Transport Laboratory},\\
            Department of Physics and Astronomy,\\
            Wayne State University\\
            Supervisor: Zhixian Zhou, Ph.D.\\
\end{innerlist}

\textbf{Undergraduate Researcher} \hfill {Jan 2014--May 2014}
\begin{innerlist}
    \item[] International Course on Computational Physics (ICCP)\\
        \href{http://www.pa.msu.edu}{Michigan State University} and \href{http://tudelft.nl/en/}{Technische Universiteit Delft}\\
        East Lansing, MI USA and Delft, Netherlands\\
        Supervisors: Phil Duxbury, Ph.D. and Jos Thijssen, Ph.D.\\
\end{innerlist}

\textbf{Undergraduate Research Assistant} \hfill {Feb 2013--Dec 2013}
\begin{innerlist}

\item[] Neutron Star Evolution and Developmental Limits,\\
        Department of Astronomy,\\
        Michigan State University\\
        Supervisor: Edward Brown, Ph.D\\
\end{innerlist}

\textbf{Undergraduate Research Assistant} \hfill {May 2012--Jan 2013}
\begin{innerlist}

\item[] \href{https://groups.nscl.msu.edu/hira/}{High Resolution Array Group (HIRA)}: SAMURAI-TPC Project\\
        \href{http://www.nscl.msu.edu/}{National Superconducting Cyclotron Laboratory},\\
        Michigan State University\\
        Supervisors: William Lynch, Ph.D. and Betty Tsang, Ph.D.\\
\end{innerlist}

\section{Industry Experience}
\textbf{Summer Intern} \hfill {Jun 2013--Aug 2013}
\begin{innerlist}
    \item[] \href{https://www.jenoptik.com/us_home}{Jenoptik Laser Technologies}, Brighton, MI USA\\
    Contributed in development of user interface for laser welding\\ machine that allows user manipulation of robotic end-arm tooling.\\ Using microcontroller program via interfaced electronic devices and\\ several developed algorithms machine was able to analyze\\
    physical data and feedback.
\end{innerlist}

\section{Publications}
\vspace{-.1275in}
\begin{bibsection}
    \item Chamlagain, B., Perera, M., Chuang, H.J., Bowman, A., Rijal, U., {\bf Andrews, K.}, Klesko, J., Winter, C., Zhou, Z. ``\href{https://scholar.google.com/citations?view_op=view_citation&hl=en&user=0FXGNzcAAAAJ&citation_for_view=0FXGNzcAAAAJ:u5HHmVD_uO8C}{Substrate dependence of Hall and Field-effect mobilities in few-layer $\mathrm{MoS}_2$ field-effect transistors}." \emph{Manuscript in preperation}, 2016.
\end{bibsection}

% Add a little space to nudge next ``Conference Publications'' marginpar
% down to make room for tall ``Submitted Journal Publications''
% marginpar. If there are enough submitted journal publications, this
% space will not be needed (and should be removed).
%\vspace{0.1in}
\section{Conference Publications}
\vspace{-.1275in}
\begin{bibsection}
    \item Chamlagain, B., Perera, M., Chuang, H.J., Bowman, A., Rijal, U., {\bf Andrews, K.}, Klesko, J., Winter, C., Zhou, Z. ``\href{https://scholar.google.com/citations?view_op=view_citation&hl=en&user=0FXGNzcAAAAJ&citation_for_view=0FXGNzcAAAAJ:u5HHmVD_uO8C}{Substrate dependence of Hall and Field-effect mobilities in few-layer $\mathrm{MoS}_2$ field-effect transistors}." Bulletin of the American Physical Society, 2016.
\end{bibsection}
%\bibliographystyle{plain}
%\bibliography{kraig_refs}
%\nocite{*}

\halfblankline


\section{Teaching Experience}

Teahcing Assistant, General Physics II, Wayne State University \hfill { Winter 2017}

Teaching Assistant, General Physics II, Wayne State University \hfill {Autumn 2016}

Teaching Assistant, General Physics I, Wayne State University \hfill {Summer 2016}

Teaching Assistant, General Physics I, Wayne State University \hfill {Autumn 2015}

Teaching Assistant, General Physics Lab I, Wayne State University \hfill {Summer 2015}

Laboratory Instructor, Conceptual Physics, Wayne State University \hfill {Winter 2015}

Laboratory Instructor, Descriptive Astronomy, Wayne State University \hfill {Winter 2015}

Laboratory Instructor, Descriptive Astronomy, Wayne State University \hfill {Autumn 2014}

Teaching Assistant, Introductory Physics II, Michigan State University \hfill {Winter 2014}

Laboratory Instructor, Planets and Telescopes, Michigan State University \hfill {Winter 2013}

Teaching Assistant, Introductory Physics I, Michigan State University
 \hfill {Autumn 2013}

Teaching Assistant, Introductory Physics II, Michigan State University \hfill {Winter 2012}

\section{Relevant Skills}

\vspace{.1275in}
Nanofabrication:
\begin{innerlist}
    \item[] Atomic Force Microscopy (AFM), Electron Beam Lithography, Photolithography, \href{http://www.autodesk.com/solutions/cad-software}{Computer-Aided Design (CAD)}, Scanning Electron Microscopy (SEM), clean room, chemical etching, metal deposition, and others
\end{innerlist}

\halfblankline

Programming:
\begin{innerlist}
    \item[] C, C$+$$+$, Fortran, \href{https://www.gnu.org/software/make/}{GNU make}, HTML, CSS, Python, UNIX shell scripting, and Visual Basic
\end{innerlist}

\halfblankline

Data Analysis:
\begin{innerlist}
    \item[] \href{https://www.gnu.org/software/octave/}{GNU octave}, \href{http://www.synergy.com/wordpress_650164087/kaleidagraph/}{Kaleidagraph}, \href{http://www.ni.com/labview/}{LabView}, \href{http://www.mathworks.com/products/matlab/?requestedDomain=www.mathworks.com}{\Matlab}, \href{https://www.wolfram.com/mathematica/}{Mathematica}, Microsoft Excel
\end{innerlist}

\halfblankline

Data Analysis:
\begin{innerlist}
    \item[] Apple OS X, Linux OS, Microsoft Windows Family
\end{innerlist}

\halfblankline

Editing and Typesetting:
\begin{innerlist}
    \item[] \href{https://www.latex-project.org/}{\TeX}, Microsoft Office, OpenOffice, LibreOffice, \href{https://www.gimp.org/}{GIMP}, \href{https://inkscape.org/en/}{InkScape}
\end{innerlist}

\halfblankline

Version Control:
\begin{innerlist}
    \item[] \href{https://github.com/}{Git}, \href{https://www.mercurial-scm.org/}{Mercurial}, \href{https://subversion.apache.org/}{SVN}
\end{innerlist}

\section{Relevant Graduate Coursework}

\begin{innerlist}
    \item[] Advanced Quantum Mechanics I \& II
    \item[] Survey of Condensed Matter Physics
    \item[] Statistical Mechanics
    \item[] Electrodynamics
    \item[] Thermal Physics
\end{innerlist}

%\end{comment}

\end{document}

%%%%%%%%%%%%%%%%%%%%%%%%%% End CV Document %%%%%%%%%%%%%%%%%%%%%%%%%%%%%

%----------------------------------------------------------------------%
% The following is copyright and licensing information for
% redistribution of this LaTeX source code; it also includes a liability
% statement. If this source code is not being redistributed to others,
% it may be omitted. It has no effect on the function of the above code.
%----------------------------------------------------------------------%
% Copyright (c) 2007, 2008, 2009, 2010, 2011 by Theodore P. Pavlic
%
% Unless otherwise expressly stated, this work is licensed under the
% Creative Commons Attribution-Noncommercial 3.0 United States License. To
% view a copy of this license, visit
% http://creativecommons.org/licenses/by-nc/3.0/us/ or send a letter to
% Creative Commons, 171 Second Street, Suite 300, San Francisco,
% California, 94105, USA.
%
% THE SOFTWARE IS PROVIDED "AS IS", WITHOUT WARRANTY OF ANY KIND, EXPRESS
% OR IMPLIED, INCLUDING BUT NOT LIMITED TO THE WARRANTIES OF
% MERCHANTABILITY, FITNESS FOR A PARTICULAR PURPOSE AND NONINFRINGEMENT.
% IN NO EVENT SHALL THE AUTHORS OR COPYRIGHT HOLDERS BE LIABLE FOR ANY
% CLAIM, DAMAGES OR OTHER LIABILITY, WHETHER IN AN ACTION OF CONTRACT,
% TORT OR OTHERWISE, ARISING FROM, OUT OF OR IN CONNECTION WITH THE
% SOFTWARE OR THE USE OR OTHER DEALINGS IN THE SOFTWARE.
%----------------------------------------------------------------------%
